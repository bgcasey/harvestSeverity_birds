% Options for packages loaded elsewhere
\PassOptionsToPackage{unicode}{hyperref}
\PassOptionsToPackage{hyphens}{url}
\PassOptionsToPackage{dvipsnames,svgnames,x11names}{xcolor}
%
\documentclass[
  12pt,
]{article}
\usepackage{amsmath,amssymb}
\usepackage{lmodern}
\usepackage{setspace}
\usepackage{iftex}
\ifPDFTeX
  \usepackage[T1]{fontenc}
  \usepackage[utf8]{inputenc}
  \usepackage{textcomp} % provide euro and other symbols
\else % if luatex or xetex
  \usepackage{unicode-math}
  \defaultfontfeatures{Scale=MatchLowercase}
  \defaultfontfeatures[\rmfamily]{Ligatures=TeX,Scale=1}
\fi
% Use upquote if available, for straight quotes in verbatim environments
\IfFileExists{upquote.sty}{\usepackage{upquote}}{}
\IfFileExists{microtype.sty}{% use microtype if available
  \usepackage[]{microtype}
  \UseMicrotypeSet[protrusion]{basicmath} % disable protrusion for tt fonts
}{}
\makeatletter
\@ifundefined{KOMAClassName}{% if non-KOMA class
  \IfFileExists{parskip.sty}{%
    \usepackage{parskip}
  }{% else
    \setlength{\parindent}{0pt}
    \setlength{\parskip}{6pt plus 2pt minus 1pt}}
}{% if KOMA class
  \KOMAoptions{parskip=half}}
\makeatother
\usepackage{xcolor}
\usepackage[margin=1in]{geometry}
\usepackage{longtable,booktabs,array}
\usepackage{calc} % for calculating minipage widths
% Correct order of tables after \paragraph or \subparagraph
\usepackage{etoolbox}
\makeatletter
\patchcmd\longtable{\par}{\if@noskipsec\mbox{}\fi\par}{}{}
\makeatother
% Allow footnotes in longtable head/foot
\IfFileExists{footnotehyper.sty}{\usepackage{footnotehyper}}{\usepackage{footnote}}
\makesavenoteenv{longtable}
\usepackage{graphicx}
\makeatletter
\def\maxwidth{\ifdim\Gin@nat@width>\linewidth\linewidth\else\Gin@nat@width\fi}
\def\maxheight{\ifdim\Gin@nat@height>\textheight\textheight\else\Gin@nat@height\fi}
\makeatother
% Scale images if necessary, so that they will not overflow the page
% margins by default, and it is still possible to overwrite the defaults
% using explicit options in \includegraphics[width, height, ...]{}
\setkeys{Gin}{width=\maxwidth,height=\maxheight,keepaspectratio}
% Set default figure placement to htbp
\makeatletter
\def\fps@figure{htbp}
\makeatother
\setlength{\emergencystretch}{3em} % prevent overfull lines
\providecommand{\tightlist}{%
  \setlength{\itemsep}{0pt}\setlength{\parskip}{0pt}}
\setcounter{secnumdepth}{5}
\newlength{\cslhangindent}
\setlength{\cslhangindent}{1.5em}
\newlength{\csllabelwidth}
\setlength{\csllabelwidth}{3em}
\newlength{\cslentryspacingunit} % times entry-spacing
\setlength{\cslentryspacingunit}{\parskip}
\newenvironment{CSLReferences}[2] % #1 hanging-ident, #2 entry spacing
 {% don't indent paragraphs
  \setlength{\parindent}{0pt}
  % turn on hanging indent if param 1 is 1
  \ifodd #1
  \let\oldpar\par
  \def\par{\hangindent=\cslhangindent\oldpar}
  \fi
  % set entry spacing
  \setlength{\parskip}{#2\cslentryspacingunit}
 }%
 {}
\usepackage{calc}
\newcommand{\CSLBlock}[1]{#1\hfill\break}
\newcommand{\CSLLeftMargin}[1]{\parbox[t]{\csllabelwidth}{#1}}
\newcommand{\CSLRightInline}[1]{\parbox[t]{\linewidth - \csllabelwidth}{#1}\break}
\newcommand{\CSLIndent}[1]{\hspace{\cslhangindent}#1}
\usepackage{booktabs}
\usepackage{lineno}
\linenumbers
\usepackage {hyperref}
\hypersetup {colorlinks = true, linkcolor = blue, urlcolor = blue}
\usepackage{multirow}
\usepackage{amsmath, amssymb} 
\usepackage{enumitem, array}
\usepackage{float}
\floatplacement{figure}{H}
\usepackage{mathtools}
\ifLuaTeX
  \usepackage{selnolig}  % disable illegal ligatures
\fi
\IfFileExists{bookmark.sty}{\usepackage{bookmark}}{\usepackage{hyperref}}
\IfFileExists{xurl.sty}{\usepackage{xurl}}{} % add URL line breaks if available
\urlstyle{same} % disable monospaced font for URLs
\hypersetup{
  pdftitle={Predicting avian response to forest harvesting using the Normalized Burn Ratio},
  pdfauthor={Brendan Casey},
  pdfkeywords={Avian, Boreal, Forestry, LiDAR},
  colorlinks=true,
  linkcolor={Maroon},
  filecolor={Maroon},
  citecolor={blue},
  urlcolor={Blue},
  pdfcreator={LaTeX via pandoc}}

\title{Predicting avian response to forest harvesting using the Normalized Burn Ratio}
\author{Brendan Casey}
\date{2023-05-01}

\begin{document}
\maketitle
\begin{abstract}
Understanding the effects of forestry activities on bird populations is essential for the conservation of boreal forest birds. Forest change detection algorithms can help by deriving continuous measures of harvest intensity and recovery from publicly available satellite imagery. We used remote sensing, point count data, acoustic monitoring tools, and mixed-effects regression models to evaluate the impact of variable retention forestry on the taxonomic and functional diversity of birds. Our findings suggest that spectral measures of harvest intensity, post-harvest recovery time, and fractional land cover variables related to low-lying vegetation and water are key drivers of post-harvest bird communities, and harvest residuals can alleviate the effects of forest harvesting on bird communities over time. Across harvest intensities, taxonomic richness, functional richness, functional dispersion, and functional evenness recovered to unharvested levels in less than 25 years post-harvest. Furthermore, we demonstrate that metrics derived from a time-series of Normalized Burn Ratio (NBR) are a promising alternative to conventional categorical harvest intensity metrics from classified land cover maps. Our study highlights the potential of remote sensing algorithms and community functional indices to uncover subtle relationships between forestry practices and bird communities.
\end{abstract}

{
\hypersetup{linkcolor=}
\setcounter{tocdepth}{2}
\tableofcontents
}
\setstretch{1.5}
\hypertarget{introduction}{%
\section{Introduction}\label{introduction}}

Canada's boreal region contains large intact forest spanning over 270 million hectares, providing critical breeding habitat and stopover sites for North American bird species that migrate to the tropics during winter months (\protect\hyperlink{ref-Blancher2005}{Blancher and Wells, 2005}; \protect\hyperlink{ref-NaturalResourcesCanada2017}{Natural Resources Canada, 2017}). Almost half of all bird species in North America rely on this region for food, shelter, and nesting sites (\protect\hyperlink{ref-wells2014boreal}{Wells et al., 2014}). However, forestry and energy exploration have altered the composition, structure, and age distribution of boreal forests, altering bird communities and populations (\protect\hyperlink{ref-Brandt2013}{Brandt et al., 2013}; \protect\hyperlink{ref-nortonSongbirdResponsePartialcut1997}{Norton and Hannon, 1997}; \protect\hyperlink{ref-Schmiegelow1997}{Schmiegelow et al., 1997}; \protect\hyperlink{ref-Venier2014}{Venier et al., 2014}). Therefore, understanding the impact of forestry activities on bird communities is crucial for the conservation of migratory birds in boreal forests.

Forestry companies are increasingly adopting sustainable practices, such as retention harvesting, to mitigate the effects of forestry on biodiversity and emulate natural disturbance regimes (\protect\hyperlink{ref-FedrowitzKoricheva2014}{Fedrowitz et al., 2014}; \protect\hyperlink{ref-galettoVariableRetentionHarvesting2019}{Galetto et al., 2019}). Variable retention forestry is used to promote multifunctional landscapes by maintaining pre-harvest legacy structures, such as patches of live standing trees, dead woody debris, and understory vegetation, throughout the harvest cycle (\protect\hyperlink{ref-Franklin2000}{Franklin et al., 2000}; \protect\hyperlink{ref-Lindenmayer2012}{Lindenmayer et al., 2012}). Retention accelerates the recovery of harvest blocks by increasing structural complexity relative to clear cuts, emulates natural disturbances like fire and insect outbreaks, and optimizes habitat for keystone or protected species (\protect\hyperlink{ref-galettoVariableRetentionHarvesting2019}{Galetto et al., 2019}; \protect\hyperlink{ref-Lindermayer2002}{Lindermayer and Franklin, 2002}; \protect\hyperlink{ref-SerrouyaR.DEon2004}{Serrouya and D'Eon, 2004}). By maintaining legacy structures, retention can improve the continuity of ecological processes and organisms across forest generations by maintaining habitat for species with low dispersal and/or small home ranges, enhancing ecological connectivity via residual stepping stones, accelerating stand recolonization by late successional species, moderating changes to microclimate, and maximizing niche availability through the maintenance of structural complexity (\protect\hyperlink{ref-BakerRead2011}{Baker and Read, 2011}; \protect\hyperlink{ref-BakerJordan2016}{Baker et al., 2016}, \protect\hyperlink{ref-bakerMicroclimateSpaceTime2014}{2014}; \protect\hyperlink{ref-ChanMcleod2007}{Chan-Mcleod and Moy, 2007}; \protect\hyperlink{ref-FedrowitzKoricheva2014}{Fedrowitz et al., 2014}; \protect\hyperlink{ref-Franklin2000}{Franklin et al., 2000}; \protect\hyperlink{ref-heitheckerEdgerelatedGradientsMicroclimate2007}{Heithecker and Halpern, 2007}; \protect\hyperlink{ref-TewsBrose2004}{Tews et al., 2004}).

Researchers have found that retention forestry can facilitate post-harvest ecological recovery (\protect\hyperlink{ref-FedrowitzKoricheva2014}{Fedrowitz et al., 2014}; \protect\hyperlink{ref-moriRetentionForestryMajor2014}{Mori and Kitagawa, 2014}). However, studies assessing the impacts of retention on bird communities have often relied on broad categorical harvest intensity metrics based on basal area or percent canopy coverage (e.g., 1-25\%, 26-50\%, 51-75\%, 75-95\%, 96-100\% disturbance) to compare the short-term (\textless15 years) effects of harvesting. While these metrics can to predict post-harvest bird communities (\protect\hyperlink{ref-odsenBorealSongbirdsVariable2018}{Odsen et al., 2018}; \protect\hyperlink{ref-priceLongtermResponseForest2020}{Price et al., 2020}), they are often obtained from intensive fieldwork or from digital land cover maps that can be slow to update, require specialized knowledge to produce, and may not capture the full range of harvest intensities present in the landscape.

Remote sensing can provide detailed, up-to-date information on the magnitude and recovery of harvests, offering a promising alternative to categorical harvest intensity metrics, without intensive field-based vegetation surveys. Continuous measures of disturbance magnitude from remote sensing may reveal subtler relationships between forestry practices and bird communities. The Landsat program has collected a continuous 50-year archive of global land surface imagery, making it well-suited for monitoring long-term vegetation change (\protect\hyperlink{ref-cohenLandsatRoleEcological2004}{Cohen and Goward, 2004}; \protect\hyperlink{ref-wulderLandsatContinuityIssues2008}{Wulder et al., 2008}). A growing body of image processing and change detection methods, along with the public availability of the full Landsat archive, are providing new ways to analyze optical time series data (\protect\hyperlink{ref-gomezOpticalRemotelySensed2016}{Gomez et al., 2016}; \protect\hyperlink{ref-tewkesburyCriticalSynthesisRemotely2015}{Tewkesbury et al., 2015}; \protect\hyperlink{ref-zhuChangeDetectionUsing2017a}{Zhu, 2017}). For example, algorithms such as Landsat-based detection of Trends in Disturbance and Recovery (LandTrendr) (\protect\hyperlink{ref-Kennedy2018}{Kennedy et al., 2018}), Breaks For Additive Seasonal and Trend (BFAST) (\protect\hyperlink{ref-verbesseltDetectingTrendSeasonal2010}{Verbesselt et al., 2010}), and the Continuous Change Detection and Classification (CCDC) (\protect\hyperlink{ref-zhuContinuousChangeDetection2014}{Zhu and Woodcock, 2014}) can produce spectral recovery and disturbance metrics using Google Earth Engine.

In this study, we used bird point count data, acoustic monitoring tools, and spectral measures of harvest intensity to examine the impact of retention forestry on bird communities over time. Specifically, we employed an annual time series of Landsat Normalized Burn Ratio (NBR) to quantify the intensity and recovery of forest harvests and used these metrics to predict the functional and species diversity of birds within harvested areas. Our objectives were threefold: (1) quantify the influence of variable retention on bird communities along a gradient of recovery, (2) assess the efficacy of NBR as a measure of harvest intensity in predictive bird models, and (3) determine and compare the time required for bird communities to return to pre-harvest conditions following different harvest treatments.

\hypertarget{methods}{%
\section{Methods}\label{methods}}

\hypertarget{study-area}{%
\subsection{Study area}\label{study-area}}

Human and acoustic point counts were conducted within harvested forest areas spread across 413,161 km\(^2\) of the Foothills and Boreal Forest Natural Regions of Alberta, Canada (Figure \ref{fig:studyArea}). The Foothills Natural Region is situated at the eastern edge of the Rocky Mountains and is dominated by mixedwood forests comprising lodgepole pine (Pinus contorta), white spruce (Picea glauca), trembling aspen (Populus tremuloides), and balsam poplar (Populus balsamifera) at lower elevations. Lodgepole pine forests are typical of higher elevations (\protect\hyperlink{ref-Downing2006}{Natural Regions Committee, 2006}). The Boreal Forest Natural Region, which spans across most of northern Alberta, comprises coniferous forests dominated by black spruce (\emph{Picea mariana}), white spruce, and jack pine (\emph{Pinus banksiana}), mixedwood forests with trembling aspen and balsam poplar, and shrubby black spruce fens (\protect\hyperlink{ref-Downing2006}{Natural Regions Committee, 2006}). Despite the differences in vegetation and topography, there is a significant overlap of bird species between the foothills and boreal. In both regions, energy production and the harvesting of aspen and conifer are common.

\begin{figure}[H]
\includegraphics[width=1\linewidth,]{../3_output/maps/studyArea_inset} \caption{Locations of point counts in the Boreal Central Mixedwood Natural Subregion of Northern Alberta.}\label{fig:studyArea}
\end{figure}

\hypertarget{forest-harvest-data}{%
\subsection{Forest harvest data}\label{forest-harvest-data}}

We used two sources of data to identify forest harvest areas: the Common Attribute Schema for Forest Resource Inventories (CAS-FRI) (\protect\hyperlink{ref-Cumming2011a}{Cosco, 2011}) and the Wall-to-Wall Human Footprint Inventory (HFI) (\protect\hyperlink{ref-abmi2019HFI}{Program, 2019}). CAS-FRI provides standardized 2 ha forest attributes derived from 1:10,000 to 1:40,000 aerial photography flown between 1987 and 2010 (\protect\hyperlink{ref-Cumming2011a}{Cosco, 2011}). HFI contains human footprint attributes interpreted from aerial and SPOT satellite imagery acquired between 1950 and 2019 (\protect\hyperlink{ref-abmi2019HFI}{Program, 2019}). We selected harvests with a range of CAS-FRI defined categorical harvest intensities (100-95, 95-75, 75-50, 50-25, and 25-0 percent).

\hypertarget{bird-data}{%
\subsection{Bird data}\label{bird-data}}

Avian point counts were conducted in harvest areas ranging from 5 to 30 years post-harvest. We used bird detection data from databases managed by the Boreal Avian Modelling Project (BAM) (\protect\hyperlink{ref-BAM2018}{Boreal Avian Modelling Project (BAM), 2018}) and the University of Alberta's Bioacoustic Unit (\protect\hyperlink{ref-bioacousticunit2021}{Unit, 2021}). Our study included data from 9700 individual point counts conducted in or within 800 meters of 1181 harvest blocks between 1995 and 2021. These point counts were conducted by human observers and autonomous recording units deployed by graduate students and field technicians. Surveys were conducted within sampling radii ranging from 50 to 150 meters and lasted between one and ten minutes. We included point counts conducted during the breeding season (May 16 to July 7) between sunrise and 10:00 h. Each location was surveyed between three and ten times approximately three days apart.

For each point count location, we computed site-level bird species richness, Shannon diversity, and functional diversity indices based on the diet, foraging, and nesting traits of species (\protect\hyperlink{ref-EltonTraits2021}{Wilman et al., 2014}) (Table \ref{tab:responseDes}). We calculated functional divergence (FDiv), functional evenness (FEve), functional richness (FRic), and functional dispersion (FDis) using the \emph{dbFD} function in the \emph{FD} package in R (\protect\hyperlink{ref-R-FD}{Laliberté et al., 2014}; \protect\hyperlink{ref-laliberteDistanceBasedFramework2010}{Laliberté and Legendre, 2010}; \protect\hyperlink{ref-masonFunctionalRichnessFunctional2005}{Mason et al., 2005}; \protect\hyperlink{ref-villegerNewMultidimensionalFunctional2008}{Villeger et al., 2008}). Shannon diversity was calculated using the \emph{diversity} function in the \emph{vegan} package in R (\protect\hyperlink{ref-R-vegan}{Oksanen et al., 2020}). Our analysis was limited to bird species with known breeding ranges in the Foothills and Boreal Forest Natural Regions of Alberta, Canada and that were observed at over three point count locations.









\begin{table}[h!]

\caption{\label{tab:responseDes}Response variables included in the analysis.}
\centering
\resizebox{\linewidth}{!}{
\fontsize{8}{10}\selectfont
\begin{tabular}[t]{l>{\raggedright\arraybackslash}p{30em}}
\toprule
Response variable & Description\\
\midrule
Functional dispersion (FDis) & The mean distance in a functional trait space between individual species and the centroid of all species present in a community weighted according relative species abundance (\protect\hyperlink{ref-laliberteDistanceBasedFramework2010}{Laliberté and Legendre, 2010})\\
Functional divergence (FDiv) & The extent to which the distribution of individual species abundances maximizes differences between functional traits found in a community (\protect\hyperlink{ref-masonFunctionalRichnessFunctional2005}{Mason et al., 2005})\\
Functional evenness (FEve) & The regularity of the distribution of abundance along the minimum spanning tree which links all species in a functional trait space (\protect\hyperlink{ref-villegerNewMultidimensionalFunctional2008}{Villeger et al., 2008})\\
Functional richness (FRic) & The minimum convex hull volume that includes all species in a functional trait space (\protect\hyperlink{ref-villegerNewMultidimensionalFunctional2008}{Villeger et al., 2008})\\
Richness & The sum of species detected over multiple visits in a season\\
Shannon diversity index (shan) & ${H}' = -\sum_i p_i \ln p_i$, where
${p_i}$ is the proportion of a community comprised of species ${i}$  (\protect\hyperlink{ref-R-vegan}{Oksanen et al., 2020})\\
\bottomrule
\end{tabular}}
\end{table}

\hypertarget{model-covariates}{%
\subsection{Model covariates}\label{model-covariates}}

A time series' of NBR can be used to assess changes in forest structure and vegetation following disturbances. NBR is a spectral index that is calculated using near-infrared (NIR) and shortwave-infrared (SWIR) reflectance bandwidths (\protect\hyperlink{ref-keyLandscapeAssessmentGround2006}{Key and Benson, 2006}). The index can be used to differentiate between undisturbed forests and harvested or burned stands. NBR is calculated by subtracting SWIR from NIR and then dividing the result by the sum of SWIR and NIR (\(NBR=\frac{NIR-SWIR}{NIR+SWIR}\))
A decrease in NBR indicates a loss of green vegetation, and the magnitude of loss can be used as a proxy for harvest intensity. When harvest areas regenerate, NBR increases with the return of green vegetation and forest structure (\protect\hyperlink{ref-hislopUsingLandsatSpectral2018}{Hislop et al., 2018}; \protect\hyperlink{ref-veraverbekeEvaluatingSpectralIndices2011}{Veraverbeke et al., 2011}; \protect\hyperlink{ref-whiteConfirmationPostharvestSpectral2018}{White et al., 2018}).

To calculate harvest intensity and recovery metrics using NBR, we used methods developed by Hird et al. (\protect\hyperlink{ref-hirdSatelliteTimeSeries2021}{2021}). First, we pre-processed harvest polygons using the \emph{sf} package in R (\protect\hyperlink{ref-R-sf}{Pebesma, 2020}). We repaired errors in polygon geometries, dissolved ``doughnut shaped'' polygons, buffered polygons by -30 m to minimize the influence of harvest edges on NBR estimates, and simplified polygons using a tolerance of 5 m. We uploaded the pre-processed harvest polygons as a shapefile asset to Google Earth Engine (\protect\hyperlink{ref-gorelickGoogleEarthEngine2017}{Gorelick et al., 2017}). Next, we generated 30 m summer (June-September) composite NBR rasters from 1984 to 2021 using images from the Landsat 5 Thematic Mapper (bands 4 and 7), Landsat 7 Enhanced Thematic Mapper (bands 4 and 7), and Landsat 8 Operational Land Imager (bands 5 and 7) via Google Earth Engine's JavaScript API (\protect\hyperlink{ref-geologicalsurveyLandsat47Surface2018}{Survey, 2018}). Snow, cloud, and cloud shadow pixels were masked and removed using the CFMask algorithm (\protect\hyperlink{ref-fogaCloudDetectionAlgorithm2017}{Foga et al., 2017}). Finally, we applied the LandTrendr algorithm to NBR composites to generate the following spectral change metrics for each forest harvest area: the mean NBR value for the five years pre-harvest (NBR\textsubscript{pre-disturbance}), the lowest NBR post-harvest value (NBR\textsubscript{post-disturbance}), NBR spectral change (\(\Delta\)NBR=NBR\textsubscript{pre-disturbance}-NBR\textsubscript{post-disturbance}), and relative spectral change (\(R\Delta\)NBR=NBR\textsubscript{pre-disturbance} - NBR\textsubscript{post-disturbance} / (\(\sqrt(|\)NBR\textsubscript{pre-disturbance}\(/1000|)\))(\protect\hyperlink{ref-MillerThode2007}{Miller and Thode, 2007})

In addition to NBR recovery metrics, for each harvest we calculated the harvest area, perimeter-area ratio of harvest boundaries, and the Euclidean distance between point count locations and the nearest unharvested forest edge. These calculations were performed using the \emph{sf} package in R (\protect\hyperlink{ref-R-sf}{Pebesma, 2020}). The mean area of individual harvests was 115.40 ha (SD=110.23). We also calculated fractional land cover for the year of the point count, mean canopy height, and the standard deviation of canopy height within a 300 m circular buffer of point count locations using Google Earth Engine (\protect\hyperlink{ref-hermosillaLandCoverClassification2022}{Hermosilla et al., 2022}; \protect\hyperlink{ref-lang2022high}{Lang et al., 2022}). Model covariates and their definitions can be found in Table \ref{tab:covDes}.





\begin{table}[h!]

\caption{\label{tab:covDes}Spatial covariates included in the analysis.}
\centering
\resizebox{\linewidth}{!}{
\fontsize{8}{10}\selectfont
\begin{tabular}[t]{ll}
\toprule
Predictor &  Description \\
\midrule
area & Area of thet harvest polygon\\
canopy\textunderscore height & Mean canopy height (\protect\hyperlink{ref-lang2022high}{Lang et al., 2022})\\
canopy\textunderscore standard\textunderscore deviation & Standard deviation of canopy height (\protect\hyperlink{ref-lang2022high}{Lang et al., 2022})\\
Coniferous & Percent of coniferous species (\protect\hyperlink{ref-hermosillaLandCoverClassification2022}{Hermosilla et al., 2022})\\
dist\textunderscore to\textunderscore edge & Distance of bird survey to nearest ungarvested edge\\
Exposed\textunderscore Barren\textunderscore land & Perent exposed barren land (\protect\hyperlink{ref-hermosillaLandCoverClassification2022}{Hermosilla et al., 2022})\\
Herbs & Perent  herbs (\protect\hyperlink{ref-hermosillaLandCoverClassification2022}{Hermosilla et al., 2022})\\
lat & Latitude of point count location\\
Mixedwood & Percent mixedwood (\protect\hyperlink{ref-hermosillaLandCoverClassification2022}{Hermosilla et al., 2022})\\
pa\textunderscore ratio & Ratio of the perimeter to the area of the harvest polygon\\
RdNBR & relative dNBR (\protect\hyperlink{ref-MillerThode2007}{Miller and Thode, 2007})\\
Shrubs & Percent shrubs (\protect\hyperlink{ref-hermosillaLandCoverClassification2022}{Hermosilla et al., 2022})\\
ss\textunderscore dst\textunderscore timelag\textunderscore nbr & The time between the harvest event and point count survey (yr)\\
Water & Percent water (\protect\hyperlink{ref-hermosillaLandCoverClassification2022}{Hermosilla et al., 2022})\\
\bottomrule
\end{tabular}}
\end{table}

\hypertarget{analyses}{%
\subsection{Analyses}\label{analyses}}

We employed mixed-effects regression models to investigate the impact of NBR harvest intensity indices on bird community metrics over time using the \emph{glmer} package in R \emph{lme4} (\protect\hyperlink{ref-batesFittingLinearMixedeffects2015}{Bates et al., 2015}). We used a Poisson generalized linear mixed model with a log link to predict species richness. To predict Shannon diversity, functional divergence, functional evenness, functional richness, and functional dispersion, we employed Gamma generalized linear mixed models with a log link. To account for differences in bird detections resulting from varying sampling effort, we used the log of sampling effort (i.e., the number of survey days multiplied by point count durations) as an offset term.

We followed the same modelling process for each response variable. First, we fit a global model that included all potential predictor variables as fixed effects, and nested random effects for harvest ID and survey year. Second, we assessed the linearity of predictor-response relationships by fitting separate models using linear, quadratic, and cubic terms. Third, we addressed multicollinearity by calculating pairwise Pearson correlation coefficients and VIF scores for all predictors, and iteratively removed highly correlated predictors from the global model. We kept only predictors with low correlation (\emph{r} \textless{} 0.5 and VIF \textless{} 3). Fourth, we used the `dredge' function from the R package \emph{MuMIn} to assess the performance of models containing combinations of the remaining predictors (\protect\hyperlink{ref-bartonMuMInMultimodelInference2020}{Bartoń, 2020}). For each model we calculated pseudo-R2 as an indicator of model fit Nakagawa and Schielzeth (\protect\hyperlink{ref-nakagawaGeneralSimpleMethod2013}{2013}){]}. The model with the lowest Akaike's Information Criterion (AIC) was selected as the top model (\protect\hyperlink{ref-burnhamModelSelectionMultimodel2002}{Burnham and Anderson, 2002}). Where models had similar AIC values (differing by less than two) we chose the one with the highest pseudo-\(R^2\). Finally, we calculated semi-partial \(R^2\) values for predictor variables using the \emph{r2beta} function from the \emph{r2glmm} R package with standardized general variances.

\hypertarget{results}{%
\section{Results}\label{results}}

Several fixed effects were common across the top models (Table \ref{tab:topModels}). Time since harvest and RdNBR were applied to all the top models, and the interaction between these two variables were included in the top models for richness, functional richness, functional divergence, functional dispersion, and functional evenness. Standard deviation of canopy height and fractional water and shrub cover were common fixed effects that were top contributors to model performance across response variables. In contrast, tree species composition and harvest polygon metrics, such as the perimeter to area ratio and harvest area, were not predictive of bird communities.

\begin{table}[h!]

\caption{\label{tab:topModels}The fixed effects and summary statistics for top models for each response variable.}
\centering
\resizebox{\linewidth}{!}{
\fontsize{8}{10}\selectfont
\begin{tabular}[t]{l>{\raggedright\arraybackslash}p{30em}rrr}
\toprule
Response variable & Fixed effects & AIC & $R^{2}c$ & $R^{2}m$\\
\midrule
Functional dispersion (FDis) & ss\textunderscore dst\textunderscore timelag\textunderscore nbr\textsuperscript{2} * RdNBR + dist\textunderscore to\textunderscore edge + Exposed\textunderscore Barren\textunderscore land & -16377.40 & 0.55 & 0.09\\
Functional divergence (FDiv) & ss\textunderscore dst\textunderscore timelag\textunderscore nbr * RdNBR + dist\textunderscore to\textunderscore edge + Shrubs + canopy\textunderscore standard\textunderscore deviation & -5993.94 & 0.44 & 0.08\\
Functional evenness (FEve) & ss\textunderscore dst\textunderscore timelag\textunderscore nbr\textsuperscript{2}:RdNBR + Shrubs +  Water  + canopy\textunderscore standard\textunderscore deviation +
lat & -6017.38 & 0.56 & 0.05\\
Functional richness (FRic) & ss\textunderscore dst\textunderscore timelag\textunderscore nbr\textsuperscript{2} * RdNBR + Herbs +  Water  +  Shrubs & -17267.78 & 0.59 & 0.15\\
Richness & ss\textunderscore dst\textunderscore timelag\textunderscore nbr\textsuperscript{2} * RdNBR + Exposed\textunderscore Barren\textunderscore land + Herbs + Shrubs +  Water + canopy\textunderscore standard\textunderscore deviation & 13866.24 & 0.40 & 0.24\\
Shannon diversity index (shan) & ss\textunderscore dst\textunderscore timelag\textunderscore nbr\textsuperscript{2} + RdNBR\textsuperscript{2} + Shrubs +Exposed\textunderscore Barren\textunderscore land  +  Water  + canopy\textunderscore standard\textunderscore deviation & 1844.12 & 0.71 & 0.22\\
\bottomrule
\end{tabular}}
\end{table}

\hypertarget{species-diversity}{%
\subsection{Species diversity}\label{species-diversity}}

Results show that species richness was negatively associated with the percentage of shrub cover, and positively associated with fractional water and herb cover, and the standard deviation of canopy height. The top model for species richness included the interaction between RdNBR and time since harvest as a fixed effect (Figure \ref{fig:dotDiv}). The interaction led to an inverted U-shaped effect curve at levels of RdNBR \textless85\% with maximum species richness occurring between 15 and 20 years post harvest. At high levels of harvest intensity \textless85\%, maximum species richness occurred between 20 and 25 years post-harvest. After 20 years post harvest, the top model predicted that species richness in all harvest intensities would converge with the mean species richness of unharvested stands. The linear term for time since harvest was the strongest predictor of species richness (\emph{b} = 0.197, SE = 0.023, \emph{p} \textless{} 0.001, semi-partial \(R^2\) = 0.037), followed by fractional cover of water (\emph{b} = 0.093, SE = 0.008, \emph{p} \textless{} 0.001, semi-partial \(R^2\) = 0.034), and the quadratic term for time since harvest (\emph{b} = -0.083, SE = 0.016, \emph{p} \textless{} 0.001, semi-partial \(R^2\) = 0.014).

Shannon diversity index decreased with the percentage of exposed barren land and shrubs, and increased with the standard deviation of canopy height and the fractional cover of water. The response of Shannon diversity to time since harvest was nonlinear and followed an inverted U-shaped effect curve, with maximum Shannon diversity occurring after \(\approx\) 20 years post-harvest for all harvest intensities. The top model for Shannon diversity included a quadratic term for RdNBR, resulting in a shallow U-shaped effect curve with a minimum harvest intensity of \(\approx\) 45\%. However, we found no significant interaction between time since harvest and RdNBR. After 20 years post-harvest, Shannon diversity in areas with harvest intensities below 75\% was predicted to converge with the mean Shannon diversity of unharvested forests. The quadratic term for time since harvest was the strongest predictor of Shannon diversity (\emph{b} = -0.046, SE = 0.015, \emph{p} \textless{} 0.001, semi-partial \(R^2\) = 0.099), followed by the proportion of water(\emph{b} = 0.038, SE = 0.004, \emph{p} \textless{} 0.001, semi-partial \(R^2\) = 0.016), the linear term for time since harvest (\emph{b} = 0.118, SE = 0.026, \emph{p} \textless{} 0.001, semi-partial \(R^2\) = 0.013), and the proportion of exposed barren land(\emph{b} = , SE = , \emph{p} \textless{} 0.001, semi-partial \(R^2\) = 0.002).

\begin{figure}[H]
\includegraphics[width=1\linewidth,]{../3_output/figures/time_nbrDist_dot_div_2023-04-25} \caption{Richness and Shannon diversity estimates for all time periods and NBR derived harvest intensities with 95$\%$  confidence intervals.}\label{fig:dotDiv}
\end{figure}

\hypertarget{functional-diversity}{%
\subsection{Functional diversity}\label{functional-diversity}}

Functional richness had a negative response to shrub cover and a positive response to fractional cover of water and herbs. Our findings revealed a mild interaction between time since harvest and RdNBR, leading to an inverted U-shaped curve at harvest intensities greater than 60\%. Peak functional richness observed between 15-20 years post-harvest (Figure \ref{fig:dotFDiv}. Functional richness approached the mean pre-harvest functional richness level between 15 and 20 years post-harvest across all harvest intensities. Among the predictors, fractional cover of water was the strongest predictor of functional richness (\emph{b} = 0.123, SE = 0.016, \emph{p} \textless{} 0.001, semi-partial \(R^2\) = 0.024), followed by the linear term of time since harvest (\emph{b} = 0.295, SE = 0.065, \emph{p} \textless{} 0.001, semi-partial \(R^2\) = 0.020), and fractional shrub cover (\emph{b} = -0.125, SE = 0.028, \emph{p} \textless{} 0.001, semi-partial \(R^2\) = 0.014).

Functional divergence decreased with time since harvest, shrub cover, and the distance to the nearest forested edge, and increased with RdNBR. The top model for functional divergence included the interaction between time since harvest and RdNBR. For harvest intensities below 50\%, functional divergence converged with the mean functional divergence of unharvested stands after 20 years. Time since harvest was the strongest predictor of functional divergence (\emph{b} = -0.021, SE = 0.007, \emph{p} \textless{} 0.001, semi-partial \(R^2\) = 0.038), followed by RdNBR (\emph{b} = 0.028, SE = 0.006, \emph{p} \textless{} 0.001, semi-partial \(R^2\) = 0.038), and the distance to the nearest forested edge (\emph{b} = -0.022, SE = 0.002, \emph{p} \textless{} 0.001, semi-partial \(R^2\) = 0.028)

Functional dispersion was found to be negatively associated with the percentage of exposed barren land and the distance to the nearest forested edge, and positively associated with RdNBR. The top model for functional dispersion included a quadratic term for time since harvest, revealing a U-shaped response curve with the highest functional dispersion occurring between 10 and 15 years post-harvest. Functional dispersion approached the levels observed in unharvested stands between 20 and 25 years post-harvest. We found no significant interaction effect between time since harvest and RdNBR on functional dispersion. The linear term of time since harvest was the strongest predictor of functional dispersion (\emph{b} = 0.070, SE = 0.017, \emph{p} \textless{} 0.001, semi-partial \(R^2\) = 0.017), followed by the quadratic term of time since harvest (\emph{b} = -0.039, SE = -.010, \emph{p} \textless{} 0.001, semi-partial \(R^2\) = 0.017), and the distance to the nearest forested edge (\emph{b} = -0.021, SE = 0.003, \emph{p} \textless{} 0.001, semi-partial \(R^2\) = 0.008)

Functional evenness responded negatively to water, shrubs, and the standard deviation of canopy height. The top model for functional evenness included an interaction between time since harvest and RdNBR which led to an inverted U-shaped curve for harvest intensities above 50\%, with maximum functional evenness observed between 15-20 years post-harvest. Functional evenness at all harvest intensities approached mean pre-harvest functional evenness between 20 and 25 years post-harvest. The interaction between the linear term of time since harvest and RdNBR was the strongest predictor of functional evenness (\emph{b} = 0.017, SE = 0.007, \emph{p} \textless{} 0.01, semi-partial \(R^2\) = 0.010), followed by the interaction between the quadratic term of time since harvest and RdNBR (\emph{b} = -0.008, SE = 0.003, \emph{p} \textless{} 0.05, semi-partial \(R^2\) = 0.005), and the standard deviation of canopy height(\emph{b} = -0.007, SE = 0.004, \emph{p} \textless{} 0.075, semi-partial \(R^2\) = 0.001).

\begin{figure}[H]
\includegraphics[width=1\linewidth,]{../3_output/figures/time_nbrDist_dot_fdiv_2023-04-25} \caption{Functional diversity estimates for all time periods and NBR derived harvest intensities with 95$\%$  confidence intervals.}\label{fig:dotFDiv}
\end{figure}

\hypertarget{discussion}{%
\section{Discussion}\label{discussion}}

Our results suggest that spectral measures of harvest intensity, post-harvest recovery time, and fractional land cover variables associated with low-lying vegetation and water are important drivers of variation in post-harvest bird communities. Our study suggests that harvest residuals can buffer the impacts of forest harvesting on birds over time. In the short term, i.e., less than five years after harvesting, we observed significant changes in bird community metrics for harvest intensities ranging from 35\% to 100\%. However, the rates of community recovery varied depending on the intensity and extent of harvesting. Our analysis revealed strong evidence that harvest residuals accelerated the recovery of bird species richness, functional evenness, and functional divergence. Additionally, across all harvest intensities, taxonomic richness, functional richness, functional dispersion, and functional evenness converged with levels of unharvested reference areas in less than 25 years.

Our study provides evidence that reducing harvest intensity (I.e., spectral change) can hasten the recovery of species diversity in post-harvest stands. We found that the response curves for both richness and Shannon diversity followed an inverted U-shape in relation to time since harvest, with immediate declines observed across all harvest intensities. However, we noted opposing trends between richness and Shannon diversity in response to harvest intensity. Specifically, between 1 and 20 years after harvest, increasing harvest intensity negatively impacted richness, but resulted in higher Shannon diversity. This finding may indicate increased evenness in the distribution of species abundances post-harvest (\protect\hyperlink{ref-hill1973diversity}{Hill, 1973}).

Retention promoted the recovery of species richness. Harvest intensities of less than 60\% led to convergence of mean richness with un-harvested reference areas within 10 years. Shannon diversity at all harvest intensities converged with non-harvested areas within 10 years. However, between 10 and 20 years post-harvest, areas with harvest intensities \textgreater75\% showed increased Shannon diversity beyond the mean Shannon diversity in un-harvested areas before eventually converging with unharvested areas 25 years post-harvest.

These findings are consistent with studies that have reported a decrease in species richness following harvests and higher levels of richness in areas with high retention compared to clear-cuts (\protect\hyperlink{ref-FedrowitzKoricheva2014}{Fedrowitz et al., 2014}; \protect\hyperlink{ref-odsenBorealSongbirdsVariable2018}{Odsen et al., 2018}; \protect\hyperlink{ref-priceLongtermResponseForest2020}{Price et al., 2020}; \protect\hyperlink{ref-Twedt2019}{Twedt, 2020}). This is likely due to harvest residuals increasing the horizontal and vertical vegetation heterogeneity of regenerating stands. Increases in vegetation heterogeneity can expand the niche space, leading to higher bird species richness (\protect\hyperlink{ref-culbert2013influence}{Culbert et al., 2013}; \protect\hyperlink{ref-TewsBrose2004}{Tews et al., 2004}).

Our study revealed complex and contrasting trends between functional diversity indices and harvest intensity over time. Specifically, we found that functional richness exhibited a similar response to that of species richness, with an immediate decline in the size of the communities' functional trait space after harvest, followed by a rapid recovery and convergence with the mean functional richness of unharvested areas after approximately 20 years However, although our models suggested a mild negative effect of harvest intensity on functional richness, this effect was negligible and did not significantly contribute to the overall model performance. Therefore, our findings suggest that the size of the post-harvest functional trait space is similar across different harvest intensities.

Similar to the results of Leaver et al. (\protect\hyperlink{ref-leaver2019response}{2019}) and Edwards et al. (\protect\hyperlink{ref-edwards2013impacts}{2013}), we observed short-term declines in functional evenness (the regularity of the abundance distribution of species with different traits (\protect\hyperlink{ref-villegerNewMultidimensionalFunctional2008}{Villeger et al., 2008})) in areas with harvest intensities greater than 50\%. This could be due to short-term increases in species dominance of cavity-nesting birds and birds that nest and forage in shrubs, as new recruits out-compete other species (\protect\hyperlink{ref-Schieck2006}{Schieck and Song, 2006}). In areas with less than 50\% harvest intensity, functional evenness did not deviate from the mean functional evenness of unharvested areas across all stages of recovery. This suggests that high amounts of harvest residuals can maintain the relative abundance of functional traits that would otherwise decline with greater harvest intensities.

Conversely, our models showed an increase in functional divergence post-harvest. Functional divergence refers to the extent to which the distribution of individual species abundances maximizes differences between functional traits (\protect\hyperlink{ref-masonFunctionalRichnessFunctional2005}{Mason et al., 2005}). We found that harvest severity had a positive linear relationship with functional divergence, providing further evidence of the increased dominance of a few functionally distinct species at higher harvest intensities. While functional divergence decreased across all harvest intensities with time, it did not fully converge with the mean functional divergence of unharvested areas within 25 years post-harvest.

Functional dispersion, a measure of community heterogeneity that estimates the mean distance of species to the centroid of all species in the functional trait space (\protect\hyperlink{ref-laliberteDistanceBasedFramework2010}{Laliberté and Legendre, 2010}), declined immediately post-harvest, followed by an inverted U-shaped response curve that peaked after 10-15 years and converged with the functional dispersion of unharvested areas after 20 years. Our models suggest that, following an initial decline, functional heterogeneity of communities increases relative to that of unharvested areas before converging after 20 years post-harvest. The increase in functional dispersion between 5 and 15 years post-harvest could be from the establishment of early seral specialists (\protect\hyperlink{ref-swanson2011forgotten}{Swanson et al., 2011}).

The differential response of community metrics to harvest severity suggests that minimum retention recommendations may depend on the community indices used and the optimal timeline for recovery. Our study found that in the first 10 years after harvesting, harvest intensities of less than 50\% reduced changes to species richness, functional richness, functional evenness, and functional dispersion. However, differences between harvest treatments shrank over time and for all harvest intensities, these metrics recovered to non-harvest levels within 25 years. With functional divergence and Shannon diversity, recovery took longer to reach baseline levels. For Shannon diversity, harvest intensities greater than 75\% did not reach unharvested levels within 25 years. Among the metrics assessed, functional divergence was the slowest to recover, and harvest intensities greater than 50\% were not predicted to reach baseline levels within 25 years post-harvest. Our findings align with a growing body of forestry research that shows that even small amounts of retention can reduce community change and accelerate post-harvest recovery (\protect\hyperlink{ref-Craig2009}{Craig and Macdonald, 2009}; \protect\hyperlink{ref-gustafssonTreeRetentionConservation2010}{Gustafsson et al., 2010}; \protect\hyperlink{ref-Halpern2012}{Halpern et al., 2012}).

Our research shows that spectral measures of disturbance derived from Landsat Time Series (LTS) data, particularly the differences in NBR between pre- and post-disturbance forests, can serve as a useful substitute for common harvest intensity metrics from ground-based stem volume and canopy cover measurements or photo-interpretation of canopy cover; metrics that are often temporally and spatially limited (\protect\hyperlink{ref-bartelsTrendsPostdisturbanceRecovery2016}{Bartels et al., 2016}; \protect\hyperlink{ref-whiteConfirmationPostharvestSpectral2018}{White et al., 2018}). For measuring harvest intensity, NBR has several advantages over other indices calculated from Landsat imagery such as the Normalized Difference Vegetation Index (NDVI), Tasseled Cap Greenness (TCG), and the Normalized Difference Moisture Index (NDMI) (\protect\hyperlink{ref-schultzPerformanceVegetationIndices2016}{Schultz et al., 2016})). First, the Short-Wave Infrared (SWIR) reflectance band used in NBR is sensitive to forest structure and vegetation moisture, making it well-suited for assessing the characteristics of regenerating stands (\protect\hyperlink{ref-hislopUsingLandsatSpectral2018}{Hislop et al., 2018}). Second, NBR has a slower saturation rate than other indices, allowing for more accurate measurement of long-term forest changes (\protect\hyperlink{ref-pickellForestRecoveryTrends2016}{Pickell et al., 2016}). And third, NBR can outperform other SWIR-based indices for forest disturbance monitoring (\protect\hyperlink{ref-cohenLandTrendrMultispectralEnsemble2018}{Cohen et al., 2018}; \protect\hyperlink{ref-veraverbekeEvaluationPrePostfire2012}{Veraverbeke et al., 2012}).

While dNBR is a useful measure of harvest severity, it does not distinguish between the various silvicultural treatments present in the landscape. Our study included a variety of treatments, such as understory protection, structural retention, dispersed single-tree retention, and aggregated retention, but we did not differentiate between these treatments when estimating harvest severity. This is a common limitation in retention studies (\protect\hyperlink{ref-gustafssonTreeRetentionConservation2010}{Gustafsson et al., 2010}; \protect\hyperlink{ref-Rosenvald2008a}{Rosenvald and Lõhmus, 2008}). However, canopy closure and the availability of large trees are closely linked to percent retention and may compensate for the lack of differentiation between treatments (\protect\hyperlink{ref-VANDERWEL2007}{Vanderwell et al., 2007}). Nevertheless, distinguishing between treatments in future research could inform managers about the optimum density and spatial configuration of retained vegetation. For example, aggregated retention may create different habitat conditions than dispersed single tree retention, influencing tree mortality, edge habitat, vegetation heterogeneity, and the persistence of mature forest and early successional bird species (\protect\hyperlink{ref-CurzonKern2020}{Curzon et al., 2020}; \protect\hyperlink{ref-VANDERWEL2007}{Vanderwell et al., 2007}).

Retention forestry should aim to provide suitable conditions for a range of species representing different functional guilds (\protect\hyperlink{ref-gustafssonRetentionForestryMaintain2012}{Gustafsson et al., 2012}). Towards this, future research should target specific federally listed species at risk or those representative of important functional guilds. Furthermore, comparing natural disturbance regimes like fire with retention forestry is crucial to understanding whether the variability of post-harvest bird communities falls within natural bounds.

\hypertarget{conclusion}{%
\section{Conclusion}\label{conclusion}}

To effectively conserve boreal birds and their habitats, it is crucial to understand the impact of forestry activities on bird populations. Our study used an annual time series of Landsat Normalized Burn Ratio to measure the intensity of forest harvesting and assess its impact on bird functional and taxonomic diversity within recovering harvested areas. While previous studies have used categorical harvest intensity metrics to assess the impacts of retention on bird communities, ours is the first to use continuous spectral measures of harvest intensity. Our findings indicate that retention forestry can mitigate the impacts of forest harvesting on bird communities. Also, including functional indices as response variables provides a more comprehensive understanding of community response compared to relying on taxonomic diversity alone (\protect\hyperlink{ref-mouillot2013functional}{Mouillot et al., 2013}). Furthermore, we demonstrated the value of LandTrendr, a cloud-based change detection algorithm, as a tool for assessing harvest intensity and recovery in boreal forests. Landsat change metrics derived using LandTrendr are useful alternatives to those from traditional classified land cover maps. The findings show that methods incorporating novel remote sensing algorithms and community functional indices can reveal subtle relationships between forestry practices and bird communities over time.

\pagebreak

\hypertarget{references}{%
\section*{References}\label{references}}
\addcontentsline{toc}{section}{References}

\hypertarget{refs}{}
\begin{CSLReferences}{1}{0}
\leavevmode\vadjust pre{\hypertarget{ref-BakerRead2011}{}}%
Baker, S.C., Read, S.M., 2011. Australian forestry variable retention silviculture in {Tasmania}'s wet forests: Ecological rationale, adaptive management and synthesis of biodiversity benefits. Australian Forestry 74, 218--232. \url{https://doi.org/10.1080/00049158.2011.10676365}

\leavevmode\vadjust pre{\hypertarget{ref-BakerJordan2016}{}}%
Baker, T.P., Jordan, G.J., Baker, S.C., 2016. Microclimatic edge effects in a recently harvested forest: {Do} remnant forest patches create the same impact as large forest areas? Forest Ecology and Management 365, 128--136. \url{https://doi.org/10.1016/j.foreco.2016.01.022}

\leavevmode\vadjust pre{\hypertarget{ref-bakerMicroclimateSpaceTime2014}{}}%
Baker, T.P., Jordan, G.J., Steel, E.A., Fountain-Jones, N.M., Wardlaw, T.J., Baker, S.C., 2014. Microclimate through space and time: {Microclimatic} variation at the edge of regeneration forests over daily, yearly and decadal time scales. Forest Ecology and Management 334, 174--184. \url{https://doi.org/10.1016/j.foreco.2014.09.008}

\leavevmode\vadjust pre{\hypertarget{ref-bartelsTrendsPostdisturbanceRecovery2016}{}}%
Bartels, S.F., Chen, H.Y.H., Wulder, M.A., White, J.C., 2016. Trends in post-disturbance recovery rates of {Canada}'s forests following wildfire and harvest. Forest Ecology and Management 361, 194--207. \url{https://doi.org/10.1016/j.foreco.2015.11.015}

\leavevmode\vadjust pre{\hypertarget{ref-bartonMuMInMultimodelInference2020}{}}%
Bartoń, K., 2020. \href{https://CRAN.R-project.org/package=MuMIn}{{MuMIn}: {Multi}-model inference} (manual).

\leavevmode\vadjust pre{\hypertarget{ref-batesFittingLinearMixedeffects2015}{}}%
Bates, D., Mächler, M., Bolker, B., Walker, S., 2015. Fitting linear mixed-effects models using {\textless{}}span class="nocase"{\textgreater{}}lme4{\textless{}}/span{\textgreater{}}. Journal of Statistical Software 67, 1--48. \url{https://doi.org/10.18637/jss.v067.i01}

\leavevmode\vadjust pre{\hypertarget{ref-Blancher2005}{}}%
Blancher, P., Wells, J., 2005. The boreal forest region: {North} {America}'s bird nursery. Canadian Boreal Initiative, Ottawa, Ontario, Canada.

\leavevmode\vadjust pre{\hypertarget{ref-BAM2018}{}}%
Boreal Avian Modelling Project (BAM), 2018. \href{http://www.borealbirds.ca/}{Boreal {Avian} {Modelling} {Project}}.

\leavevmode\vadjust pre{\hypertarget{ref-Brandt2013}{}}%
Brandt, J.P., Flannigan, M.D., Maynard, D.G., Thompson, I.D., Volney, W.J.A., 2013. An introduction to {Canada}'s boreal zone: {Ecosystem} processes, health, sustainability, and environmental issues. Environmental Reviews 21, 207--226. \url{https://doi.org/10.1139/er-2013-0040}

\leavevmode\vadjust pre{\hypertarget{ref-burnhamModelSelectionMultimodel2002}{}}%
Burnham, K.P., Anderson, D.R., 2002. Model selection and multimodel inference: {A} practical information-theoretic approach. Springer New York.

\leavevmode\vadjust pre{\hypertarget{ref-ChanMcleod2007}{}}%
Chan-Mcleod, A.C.A., Moy, A., 2007. Evaluating residual tree patches as stepping stones and short-term refugia for red-legged frogs. Journal of Wildlife Management 71, 1836--1844. \url{https://doi.org/10.2193/2006-309}

\leavevmode\vadjust pre{\hypertarget{ref-cohenLandsatRoleEcological2004}{}}%
Cohen, W.B., Goward, S.N., 2004. Landsat's role in ecological applications of remote sensing. Bioscience 54, 535--545. \url{https://doi.org/10.1641/0006-3568(2004)054\%5B0535:LRIEAO\%5D2.0.CO;2}

\leavevmode\vadjust pre{\hypertarget{ref-cohenLandTrendrMultispectralEnsemble2018}{}}%
Cohen, W.B., Yang, Z., Healey, S.P., Kennedy, R.E., Gorelick, N., 2018. A {LandTrendr} multispectral ensemble for forest disturbance detection. Remote sensing of environment 205, 131--140.

\leavevmode\vadjust pre{\hypertarget{ref-Cumming2011a}{}}%
Cosco, J.A., 2011. Common {Attribute} schema ({CAS}) for forest inventories across {Canada}. Boreal Avian Modelling Project; Canadian BEACONs Project, Edmonton, Alberta, Canada.

\leavevmode\vadjust pre{\hypertarget{ref-Craig2009}{}}%
Craig, A., Macdonald, S.E., 2009. Threshold effects of variable retention harvesting on understory plant communities in the boreal mixedwood forest. Forest Ecology and Management 258, 2619--2627. \url{https://doi.org/10.1016/j.foreco.2009.09.019}

\leavevmode\vadjust pre{\hypertarget{ref-culbert2013influence}{}}%
Culbert, P.D., Radeloff, V.C., Flather, C.H., Kellndorfer, J.M., Rittenhouse, C.D., Pidgeon, A.M., 2013. The influence of vertical and horizontal habitat structure on nationwide patterns of avian biodiversity. The Auk 130, 656--665.

\leavevmode\vadjust pre{\hypertarget{ref-CurzonKern2020}{}}%
Curzon, M.T., Kern, C.C., Baker, S.C., Palik, B.J., D'Amato, A.W., 2020. Retention forestry influences understory diversity and functional identity. Ecological Applications eap.2097. \url{https://doi.org/10.1002/eap.2097}

\leavevmode\vadjust pre{\hypertarget{ref-edwards2013impacts}{}}%
Edwards, F.A., Edwards, D.P., Hamer, K.C., Davies, R.G., 2013. Impacts of logging and conversion of rainforest to oil palm on the functional diversity of birds in {S} undaland. Ibis 155, 313--326.

\leavevmode\vadjust pre{\hypertarget{ref-FedrowitzKoricheva2014}{}}%
Fedrowitz, K., Koricheva, J., Baker, S.C., Lindenmayer, D.B., Palik, B., Rosenvald, R., Beese, W., Franklin, J.F., Kouki, J., Macdonald, E., Messier, C., Sverdrup-Thygeson, A., Gustafsson, L., 2014. Can retention forestry help conserve biodiversity? {A} meta-analysis. Journal of Applied Ecology 51, 1669--1679. \url{https://doi.org/10.1111/1365-2664.12289}

\leavevmode\vadjust pre{\hypertarget{ref-fogaCloudDetectionAlgorithm2017}{}}%
Foga, S., Scaramuzza, P.L., Guo, S., Zhu, Z., Dilley, R.D., Jr., Beckmann, T., Schmidt, G.L., Dwyer, J.L., Hughes, M.J., Laue, B., 2017. Cloud detection algorithm comparison and validation for operational {Landsat} data products. Remote Sensing of Environment 194, 379--390. \url{https://doi.org/10.1016/j.rse.2017.03.026}

\leavevmode\vadjust pre{\hypertarget{ref-Franklin2000}{}}%
Franklin, J.F., Lindenmayer, D.B., MacMahon, J.A., McKee, A., Magnusson, J., Perry, D.A., Waide, R., Foster, D.R., 2000. \href{https://www.researchgate.net/profile/David_Perry6/publication/249472082_Threads_of_Continuity._There_are_immense_differences_between_even-aged_silvicultural_disturbances_\%28es}{Threads of continuity: {Ecosystem} disturbances, biological legacies and ecosystem recovery}. Conservation Biology in Practice 1, 8--16.

\leavevmode\vadjust pre{\hypertarget{ref-galettoVariableRetentionHarvesting2019}{}}%
Galetto, L., Torres, C., Martínez Pastur, G.J., 2019. Variable retention harvesting: Conceptual analysis according to different environmental ethics and forest valuation. Ecological Processes 8. \url{https://doi.org/10.1186/s13717-019-0195-3}

\leavevmode\vadjust pre{\hypertarget{ref-gomezOpticalRemotelySensed2016}{}}%
Gomez, C., White, J.C., Wulder, M.A., 2016. Optical remotely sensed time series data for land cover classification: {A} review. ISPRS Journal of Photogrammetry and Remote Sensing 116, 55--72. \url{https://doi.org/10.1016/j.isprsjprs.2016.03.008}

\leavevmode\vadjust pre{\hypertarget{ref-gorelickGoogleEarthEngine2017}{}}%
Gorelick, N., Hancher, M., Dixon, M., Ilyushchenko, S., Thau, D., Moore, R., 2017. Google {Earth} {Engine}: {Planetary}-scale geospatial analysis for everyone. Remote sensing of Environment 202, 18--27. \url{https://doi.org/10.1016/j.rse.2017.06.031}

\leavevmode\vadjust pre{\hypertarget{ref-gustafssonRetentionForestryMaintain2012}{}}%
Gustafsson, L., Baker, S.C., Bauhus, J., Beese, W.J., Brodie, A., Kouki, J., Lindenmayer, D.B., Lõhmus, A., Pastur, G.M., Messier, C., Neyland, M., Palik, B., Sverdrup-Thygeson, A., Volney, W.J.A., Wayne, A., Franklin, J.F., 2012. Retention {Forestry} to {Maintain} {Multifunctional} {Forests}: {A} {World} {Perspective}. BioScience 62, 633--645. \url{https://doi.org/10.1525/bio.2012.62.7.6}

\leavevmode\vadjust pre{\hypertarget{ref-gustafssonTreeRetentionConservation2010}{}}%
Gustafsson, L., Kouki, J., Sverdrup-Thygeson, A., 2010. Tree retention as a conservation measure in clear-cut forests of northern {Europe}: {A} review of ecological consequences. Scandinavian Journal of Forest Research 25, 295--308. \url{https://doi.org/10.1080/02827581.2010.497495}

\leavevmode\vadjust pre{\hypertarget{ref-Halpern2012}{}}%
Halpern, C.B., Halaj, J., Evans, S.A., Dovč Iak, M., 2012. Level and pattern of overstory retention interact to shape long-term responses of understories to timber harvest. Ecological Applications 22, 2049--2064. \url{https://doi.org/10.1890/12-0299.1}

\leavevmode\vadjust pre{\hypertarget{ref-heitheckerEdgerelatedGradientsMicroclimate2007}{}}%
Heithecker, T.D., Halpern, C.B., 2007. Edge-related gradients in microclimate in forest aggregates following structural retention harvests in western {Washington}. Forest Ecology and Management 248, 163--173. \url{https://doi.org/10.1016/j.foreco.2007.05.003}

\leavevmode\vadjust pre{\hypertarget{ref-hermosillaLandCoverClassification2022}{}}%
Hermosilla, T., Wulder, M.A., White, J.C., Coops, N.C., 2022. Land cover classification in an era of big and open data: {Optimizing} localized implementation and training data selection to improve mapping outcomes. Remote Sensing of Environment 268, 112780.

\leavevmode\vadjust pre{\hypertarget{ref-hill1973diversity}{}}%
Hill, M.O., 1973. Diversity and evenness: A unifying notation and its consequences. Ecology 54, 427--432.

\leavevmode\vadjust pre{\hypertarget{ref-hirdSatelliteTimeSeries2021}{}}%
Hird, J.N., Kariyeva, J., McDermid, G.J., 2021. Satellite time series and {Google} {Earth} {Engine} democratize the process of forest-recovery monitoring over large areas. Remote Sensing 13. \url{https://doi.org/10.3390/rs13234745}

\leavevmode\vadjust pre{\hypertarget{ref-hislopUsingLandsatSpectral2018}{}}%
Hislop, S., Jones, S., Soto-Berelov, M., Skidmore, A., Haywood, A., Nguyen, T.H., 2018. Using landsat spectral indices in time-series to assess wildfire disturbance and recovery. Remote Sensing 10, 460. \url{https://doi.org/10.3390/rs10030460}

\leavevmode\vadjust pre{\hypertarget{ref-Kennedy2018}{}}%
Kennedy, R.E., Yang, Z., Gorelick, N., Braaten, J., Cavalcante, L., Cohen, W.B., Healey, S., 2018. Implementation of the {LandTrendr} algorithm on {Google} {Earth} {Engine}. Remote Sensing 10, 691. \url{https://doi.org/10.3390/rs10050691}

\leavevmode\vadjust pre{\hypertarget{ref-keyLandscapeAssessmentGround2006}{}}%
Key, C., Benson, N., 2006. Landscape assessment: Ground measure of severity, the composite burn index; and remote sensing of severity, the normalized burn ratio. USDA Forest Service.

\leavevmode\vadjust pre{\hypertarget{ref-laliberteDistanceBasedFramework2010}{}}%
Laliberté, E., Legendre, P., 2010. A distance‐based framework for measuring functional diversity from multiple traits. Ecology 91, 299--305.

\leavevmode\vadjust pre{\hypertarget{ref-R-FD}{}}%
Laliberté, E., Legendre, P., Shipley, B., 2014. \href{https://CRAN.R-project.org/package=FD}{{FD}: {Measuring} functional diversity ({FD}) from multiple traits, and other tools for functional ecology} (manual).

\leavevmode\vadjust pre{\hypertarget{ref-lang2022high}{}}%
Lang, N., Jetz, W., Schindler, K., Wegner, J.D., 2022. A high-resolution canopy height model of the {Earth}. arXiv preprint arXiv:2204.08322.

\leavevmode\vadjust pre{\hypertarget{ref-leaver2019response}{}}%
Leaver, J., Mulvaney, J., Smith, D.A.E., Smith, Y.C.E., Cherry, M.I., 2019. Response of bird functional diversity to forest product harvesting in the {Eastern} {Cape}, {South} {Africa}. Forest Ecology and Management 445, 82--95.

\leavevmode\vadjust pre{\hypertarget{ref-Lindenmayer2012}{}}%
Lindenmayer, D.B., Franklin, J.F., Lõhmus, A., Baker, S.C., Bauhus, J., Beese, W., Brodie, A., Kiehl, B., Kouki, J., Pastur, G.M., Messier, C., Neyland, M., Palik, B., Sverdrup-Thygeson, A., Volney, J., Wayne, A., Gustafsson, L., 2012. A major shift to the retention approach for forestry can help resolve some global forest sustainability issues. Conservation Letters 5, 421--431. \url{https://doi.org/10.1111/j.1755-263X.2012.00257.x}

\leavevmode\vadjust pre{\hypertarget{ref-Lindermayer2002}{}}%
Lindermayer, D.B., Franklin, J.F., 2002. Conserving forest biodiversity: A comprehensive multiscaled approach. Island Press.

\leavevmode\vadjust pre{\hypertarget{ref-masonFunctionalRichnessFunctional2005}{}}%
Mason, N.W., Mouillot, D., Lee, W.G., Wilson, J.B., 2005. Functional richness, functional evenness and functional divergence: The primary components of functional diversity. Oikos 111, 112--118. \url{https://doi.org/10.1111/j.0030-1299.2005.13886.x}

\leavevmode\vadjust pre{\hypertarget{ref-MillerThode2007}{}}%
Miller, J.D., Thode, A.E., 2007. Quantifying burn severity in a heterogeneous landscape with a relative version of the delta {Normalized} {Burn} {Ratio} ({dNBR}). Remote Sensing of Environment 109, 66--80. \url{https://doi.org/10.1016/j.rse.2006.12.006}

\leavevmode\vadjust pre{\hypertarget{ref-moriRetentionForestryMajor2014}{}}%
Mori, A.S., Kitagawa, R., 2014. Retention forestry as a major paradigm for safeguarding forest biodiversity in productive landscapes: {A} global meta-analysis. Biological Conservation 175, 65--73. \url{https://doi.org/10.1016/j.biocon.2014.04.016}

\leavevmode\vadjust pre{\hypertarget{ref-mouillot2013functional}{}}%
Mouillot, D., Graham, N.A., Villéger, S., Mason, N.W., Bellwood, D.R., 2013. A functional approach reveals community responses to disturbances. Trends in ecology \& evolution 28, 167--177.

\leavevmode\vadjust pre{\hypertarget{ref-nakagawaGeneralSimpleMethod2013}{}}%
Nakagawa, S., Schielzeth, H., 2013. A general and simple method for obtaining {R2} from generalized linear mixed-effects models. Methods in Ecology and Evolution 4, 133--142. \url{https://doi.org/10.1111/j.2041-210x.2012.00261.x}

\leavevmode\vadjust pre{\hypertarget{ref-Downing2006}{}}%
Natural Regions Committee, 2006. Natural regions and subregions of {Alberta}. Compiled by DJ Downing; WW Pettapiece. Government of Alberta, Edmonton, Alberta, Canada.

\leavevmode\vadjust pre{\hypertarget{ref-NaturalResourcesCanada2017}{}}%
Natural Resources Canada, 2017. The {State} of {Canada}'s {Forests}: {Annual} {Report} 2017.

\leavevmode\vadjust pre{\hypertarget{ref-nortonSongbirdResponsePartialcut1997}{}}%
Norton, M.R., Hannon, S.J., 1997. Songbird response to partial-cut logging in the boreal mixedwood forest of {Alberta}. Canadian Journal of Forest Research 27, 44--53. \url{https://doi.org/10.1139/x96-149}

\leavevmode\vadjust pre{\hypertarget{ref-odsenBorealSongbirdsVariable2018}{}}%
Odsen, S.G., Pinzon, J., Schmiegelow, F.K.A., Acorn, J.H., Spence, J.R., 2018. Boreal songbirds and variable retention management: A 15-year perspective on avian conservation and forestry. Canadian Journal of Forest Research 48, 1495--1502. \url{https://doi.org/10.1139/cjfr-2018-0203}

\leavevmode\vadjust pre{\hypertarget{ref-R-vegan}{}}%
Oksanen, J., Blanchet, F.G., Friendly, M., Kindt, R., Legendre, P., McGlinn, D., Minchin, P.R., O'Hara, R.B., Simpson, G.L., Solymos, P., Stevens, M.H.H., Szoecs, E., Wagner, H., 2020. \href{https://CRAN.R-project.org/package=vegan}{Vegan: {Community} ecology package} (manual).

\leavevmode\vadjust pre{\hypertarget{ref-R-sf}{}}%
Pebesma, E., 2020. \href{https://CRAN.R-project.org/package=sf}{Sf: {Simple} features for r} (manual).

\leavevmode\vadjust pre{\hypertarget{ref-pickellForestRecoveryTrends2016}{}}%
Pickell, P.D., Hermosilla, T., Frazier, R.J., Coops, N.C., Wulder, M.A., 2016. Forest recovery trends derived from {Landsat} time series for {North} {American} boreal forests. International Journal of Remote Sensing 37, 138--149. \url{https://doi.org/10.1080/2150704X.2015.1126375}

\leavevmode\vadjust pre{\hypertarget{ref-priceLongtermResponseForest2020}{}}%
Price, K., Daust, K., Lilles, E., Roberts, A.M., 2020. Long-term response of forest bird communities to retention forestry in northern temperate coniferous forests. Forest Ecology and Management 462. \url{https://doi.org/10.1016/j.foreco.2020.117982}

\leavevmode\vadjust pre{\hypertarget{ref-abmi2019HFI}{}}%
Program, A.B.M.I. and A.H.F.M., 2019. Alberta {Biodiversity} {Monitoring} {Institute} and {Alberta} {Human} {Footprint} {Monitoring} {Program} {Wall}-to-{Wall} {Human} {Footprint} {Inventory} 2019.

\leavevmode\vadjust pre{\hypertarget{ref-Rosenvald2008a}{}}%
Rosenvald, R., Lõhmus, A., 2008. For what, when, and where is green-tree retention better than clear-cutting? {A} review of the biodiversity aspects. Forest Ecology and Management 255, 1--15. \url{https://doi.org/10.1016/j.foreco.2007.09.016}

\leavevmode\vadjust pre{\hypertarget{ref-Schieck2006}{}}%
Schieck, J., Song, S.J., 2006. Changes in bird communities throughout succession following fire and harvest in boreal forests of western {North} {America}: {Literature} review and meta-analyses. Canadian Journal of Forest Research 36, 1299--1318. \url{https://doi.org/10.1139/X06-017}

\leavevmode\vadjust pre{\hypertarget{ref-Schmiegelow1997}{}}%
Schmiegelow, F.K., Machtans, C., Hannon, S., 1997. Are boreal birds resilient to forest fragmentation? {An} experimental study of short-term community responses. Ecology 78, 1914--1932. \url{https://doi.org/10.1890/0012-9658(1997)078\%5B1914:abbrtf\%5D2.0.co;2}

\leavevmode\vadjust pre{\hypertarget{ref-schultzPerformanceVegetationIndices2016}{}}%
Schultz, M., Clevers, J.G., Carter, S., Verbesselt, J., Avitabile, V., Quang, H.V., Herold, M., 2016. Performance of vegetation indices from {Landsat} time series in deforestation monitoring. International journal of applied earth observation and geoinformation 52, 318--327.

\leavevmode\vadjust pre{\hypertarget{ref-SerrouyaR.DEon2004}{}}%
Serrouya, R., D'Eon, R., 2004. Variable retention forest harvesting: Research synthesis and implementation guidelines. Sustainable Forest Management Network,.

\leavevmode\vadjust pre{\hypertarget{ref-geologicalsurveyLandsat47Surface2018}{}}%
Survey, U.S.G., 2018. Landsat 4-7 {Surface} {Reflectance} ({Ledaps}) {Product} {Guide}.

\leavevmode\vadjust pre{\hypertarget{ref-swanson2011forgotten}{}}%
Swanson, M.E., Franklin, J.F., Beschta, R.L., Crisafulli, C.M., DellaSala, D.A., Hutto, R.L., Lindenmayer, D.B., Swanson, F.J., 2011. The forgotten stage of forest succession: Early-successional ecosystems on forest sites. Frontiers in Ecology and the Environment 9, 117--125.

\leavevmode\vadjust pre{\hypertarget{ref-tewkesburyCriticalSynthesisRemotely2015}{}}%
Tewkesbury, A.P., Comber, A.J., Tate, N.J., Lamb, A., Fisher, P.F., 2015. A critical synthesis of remotely sensed optical image change detection techniques. Remote Sensing of Environment 160, 1--14. \url{https://doi.org/10.1016/j.rse.2015.01.006}

\leavevmode\vadjust pre{\hypertarget{ref-TewsBrose2004}{}}%
Tews, J., Brose, U., Grimm, V., Tielbörger, K., Wichmann, M.C., Schwager, M., Jeltsch, F., 2004. Animal species diversity driven by habitat heterogeneity/diversity: The importance of keystone structures. Journal of Biogeography 31, 79--92. \url{https://doi.org/10.1046/j.0305-0270.2003.00994.x}

\leavevmode\vadjust pre{\hypertarget{ref-Twedt2019}{}}%
Twedt, D.J., 2020. Influence of forest harvest severity and time since perturbation on conservation of {North} {American} birds. Forest Ecology and Management 458, 117742. \url{https://doi.org/10.1016/j.foreco.2019.117742}

\leavevmode\vadjust pre{\hypertarget{ref-bioacousticunit2021}{}}%
Unit, B., 2021. University of Alberta \& Alberta Biodiversity Monitoring Institute, Edmonton, Alberta.

\leavevmode\vadjust pre{\hypertarget{ref-VANDERWEL2007}{}}%
Vanderwell, M.C., Malcolm, J.R., Mills, S.C., 2007. A meta-analysis of bird responses to uniform partial harvesting across {North} {America}. Conservation Biology 21, 1230--1240. \url{https://doi.org/10.1111/j.1523-1739.2007.00756.x}

\leavevmode\vadjust pre{\hypertarget{ref-Venier2014}{}}%
Venier, L.A., Thompson, I.D., Fleming, R., Malcolm, J., Aubin, I., Trofymow, J.A., Langor, D., Sturrock, R., Patry, C., Outerbridge, R.O., Holmes, S.B., Haeussler, S., Grandpré, L.D., Chen, H.Y.H., Bayne, E., De Grandpré, L., Chen, H.Y.H., Bayne, E., Arsenault, A., Brandt, J.P., 2014. Effects of natural resource development on the terrestrial biodiversity of {Canadian} boreal forests. Environmental Reviews 22, 457--490. \url{https://doi.org/10.1139/er-2013-0075}

\leavevmode\vadjust pre{\hypertarget{ref-veraverbekeEvaluatingSpectralIndices2011}{}}%
Veraverbeke, S., Harris, S., Hook, S., 2011. Evaluating spectral indices for burned area discrimination using {MODIS}/{ASTER} ({MASTER}) airborne simulator data. Remote Sensing of Environment 115, 2702--2709. \url{https://doi.org/10.1016/j.rse.2011.06.010}

\leavevmode\vadjust pre{\hypertarget{ref-veraverbekeEvaluationPrePostfire2012}{}}%
Veraverbeke, S., Lhermitte, S., Verstraeten, W.W., Goossens, R., 2012. Evaluation of pre/post-fire differenced spectral indices for assessing burn severity in a {Mediterranean} environment with {Landsat} {Thematic} {Mapper}. International Journal of Remote Sensing 32, 3521--3537.

\leavevmode\vadjust pre{\hypertarget{ref-verbesseltDetectingTrendSeasonal2010}{}}%
Verbesselt, J., Hyndman, R., Newnham, G., Culvenor, D., 2010. Detecting trend and seasonal changes in satellite image time series. Remote Sensing of Environment 114, 106--115. \url{https://doi.org/10.1016/j.rse.2009.08.014}

\leavevmode\vadjust pre{\hypertarget{ref-villegerNewMultidimensionalFunctional2008}{}}%
Villeger, S., Mason, N.W.H., Mouillot, D., 2008. New multidimensional functional diversity indices for a multifaceted framework in functional ecology. Ecology 89, 2290--2301. \url{https://doi.org/10.1890/07-1206.1}

\leavevmode\vadjust pre{\hypertarget{ref-wells2014boreal}{}}%
Wells, J., Childs, D., Reid, F., Smith, K., Darveau, M., Courtois, V., 2014. Boreal birds need half: {Maintaining} north america's bird nursery and why it matters. {Boreal} songbird initiative, seattle, {WA}. Ducks Unlimited Inc., Memphis, TN, and Ducks Unlimited Canada, Stonewall, MB, Canada. Available online at www. borealbirds. org/sites/default/files/publications/BBNH-Report-Web. pdf.

\leavevmode\vadjust pre{\hypertarget{ref-whiteConfirmationPostharvestSpectral2018}{}}%
White, J.C., Saarinen, N., Kankare, V., Wulder, M.A., Hermosilla, T., Coops, N.C., Pickell, P.D., Holopainen, M., Hyyppä, J., Vastaranta, M., 2018. Confirmation of post-harvest spectral recovery from {Landsat} time series using measures of forest cover and height derived from airborne laser scanning data. Remote Sensing of Environment 216, 262--275. \url{https://doi.org/10.1016/j.rse.2018.07.004}

\leavevmode\vadjust pre{\hypertarget{ref-EltonTraits2021}{}}%
Wilman, H., Belmaker, J., Simpson, J., Rosa, C. de la, Rivadeneira, M.M., Jetz, W., 2014. {EltonTraits} 1.0: {Species}-level foraging attributes of the world's birds and mammals. Ecology 95, 2027--2027. \url{https://doi.org/10.1890/13-1917.1}

\leavevmode\vadjust pre{\hypertarget{ref-wulderLandsatContinuityIssues2008}{}}%
Wulder, M.A., White, J.C., Goward, S.N., Masek, J.G., Irons, J.R., Herold, M., Cohen, W.B., Loveland, T.R., Woodcock, C.E., 2008. Landsat continuity: {Issues} and opportunities for land cover monitoring. Remote Sensing of Environment 112, 955--969. \url{https://doi.org/10.1016/j.rse.2007.07.004}

\leavevmode\vadjust pre{\hypertarget{ref-zhuChangeDetectionUsing2017a}{}}%
Zhu, Z., 2017. Change detection using landsat time series: {A} review of frequencies, preprocessing, algorithms, and applications. ISPRS Journal of Photogrammetry and Remote Sensing 130, 370--384. \url{https://doi.org/10.1016/j.isprsjprs.2017.06.013}

\leavevmode\vadjust pre{\hypertarget{ref-zhuContinuousChangeDetection2014}{}}%
Zhu, Z., Woodcock, C.E., 2014. Continuous change detection and classification of land cover using all available {Landsat} data. Remote Sensing of Environment 144, 152--171. \url{https://doi.org/10.1016/j.rse.2014.01.011}

\end{CSLReferences}

\pagebreak

\end{document}
